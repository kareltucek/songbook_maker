\title{Test Printing Template}
\author{Me}
\newcommand{\introduction}{
  \maketitle
%  \tableofcontents

  This sheet serves for testing of printing offsets for "compact" styles. This is needed since these styles have reduced page margin (only 4mm). Therefore, everything needs to be printed and cut precisely for binding purposes!

  The steps needed for ensuring correct print are the following. We assume that the booklet is printed in a specialized company (any copy centre) onto A4, cut in half and bind into standard ring binding. Binding offset is 6mm, with another 4mm margin. This means that if page is properly cut, there is 10mm left for binding. 

  \begin{enumerate}
    \item First, notice four lines indicated corners of original A5 pages. They are the important part of this sheet. 
    \item Print this sheet. Try to print it borderless, if anything does not work, let the software fit it to printing area.
    \item Check that:
      \begin{itemize}
        \item Check that full text of songs is printed (otherwise you will notice some headline cut).
        \item Hold the paper against light and ensure that centerlines match on both sides of the paper.
        \item Fold the paper in (exact!) half and check that centerline is in center (unless you print borderless, it may be shifted!). If not, you will need to explain to whomever will be cutting it that it needs to be cut at centerline.
        \item Illustration is not missing. Some printers throw pages with high DPI pictures out.
      \end{itemize}
    \item Proceed with printing of the songbook.
  \end{enumerate}

  In case anything goes wrong and you need to repring with bigger binding offset, you need to edit template-a5-doublesided.tex and modify usepackage geometry line. Margin applies on all sides of every a5 page, while bindingoffset applies only on the inner side. This means that as long as "bindingoffset + 2 * margin = 14", you have guarantee that formatting remains the same (if song did fit page before, it will fit it after).

  Note that this sheet is not autogenerated and therefore edits of the file won't change the template (but that's ok, you should not needed).

  \clearpage
}

